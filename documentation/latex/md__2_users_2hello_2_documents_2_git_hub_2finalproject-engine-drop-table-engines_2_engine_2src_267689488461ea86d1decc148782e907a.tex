\chapter{undici-\/types}
\hypertarget{md__2_users_2hello_2_documents_2_git_hub_2finalproject-engine-drop-table-engines_2_engine_2src_267689488461ea86d1decc148782e907a}{}\label{md__2_users_2hello_2_documents_2_git_hub_2finalproject-engine-drop-table-engines_2_engine_2src_267689488461ea86d1decc148782e907a}\index{undici-\/types@{undici-\/types}}
\label{md__2_users_2hello_2_documents_2_git_hub_2finalproject-engine-drop-table-engines_2_engine_2src_267689488461ea86d1decc148782e907a_autotoc_md10408}%
\Hypertarget{md__2_users_2hello_2_documents_2_git_hub_2finalproject-engine-drop-table-engines_2_engine_2src_267689488461ea86d1decc148782e907a_autotoc_md10408}%


This package is a dual-\/publish of the \href{https://www.npmjs.com/package/undici}{\texttt{ undici}} library types. The {\ttfamily undici} package {\bfseries{still contains types}}. This package is for users who {\itshape only} need undici types (such as for {\ttfamily @types/node}). It is published alongside every release of {\ttfamily undici}, so you can always use the same version.


\begin{DoxyItemize}
\item \href{https://github.com/nodejs/undici}{\texttt{ Git\+Hub nodejs/undici}}
\item \href{https://undici.nodejs.org/\#/}{\texttt{ Undici Documentation}} 
\end{DoxyItemize}