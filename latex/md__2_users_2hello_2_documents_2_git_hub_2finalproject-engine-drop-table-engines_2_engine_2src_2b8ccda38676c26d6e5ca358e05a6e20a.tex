\chapter{abbrev-\/js}
\hypertarget{md__2_users_2hello_2_documents_2_git_hub_2finalproject-engine-drop-table-engines_2_engine_2src_2b8ccda38676c26d6e5ca358e05a6e20a}{}\label{md__2_users_2hello_2_documents_2_git_hub_2finalproject-engine-drop-table-engines_2_engine_2src_2b8ccda38676c26d6e5ca358e05a6e20a}\index{abbrev-\/js@{abbrev-\/js}}
\label{md__2_users_2hello_2_documents_2_git_hub_2finalproject-engine-drop-table-engines_2_engine_2src_2b8ccda38676c26d6e5ca358e05a6e20a_autotoc_md1799}%
\Hypertarget{md__2_users_2hello_2_documents_2_git_hub_2finalproject-engine-drop-table-engines_2_engine_2src_2b8ccda38676c26d6e5ca358e05a6e20a_autotoc_md1799}%


Just like \href{http://apidock.com/ruby/Abbrev}{\texttt{ ruby\textquotesingle{}s Abbrev}}.

Usage\+: \begin{DoxyVerb}var abbrev = require("abbrev");
abbrev("foo", "fool", "folding", "flop");

// returns:
{ fl: 'flop'
, flo: 'flop'
, flop: 'flop'
, fol: 'folding'
, fold: 'folding'
, foldi: 'folding'
, foldin: 'folding'
, folding: 'folding'
, foo: 'foo'
, fool: 'fool'
}
\end{DoxyVerb}


This is handy for command-\/line scripts, or other cases where you want to be able to accept shorthands. 